\documentclass[hreffootnote]{zah}

\title{Evidencia plynových fliaš}
\author{Oliver Laštík, Šimon Strieška, Jozef Špirka, Adam Zahradník}

\begin{document}
\maketitle

\tableofcontents
\cleardoublepage

\section{Úvod}

\subsection{Účel tohto dokumentu}

Tento dokument slúži ako komplexný opis systému, ktorého cieľom je poskytnúť súbor požiadaviek pre navrhovaný systém. Slúži vývojovému tímu ako jednotné miesto, v ktorom sú spísané všetky požiadavky na systém. Zároveň slúži pre zadávateľa pre prehľad dohodnutých požiadavok a kontrolu ich naplnenia.

\subsection{Rozsah pôsobnosti systému}

Systém je určený na evidenciu používaných plynových fliaš, zaznamenávanie ich umiestnenia vo forme skladovej evidencie a sledovanie spotrebu plynov prostredníctvom "odpisovania" tlaku vo flašiach.

\subsection{Používané definície a skratky}

\begin{enumerate}
	\item manometer: prístroj na meranie tlaku, ciferník s ručičkou
\end{enumerate}

\subsection{Odkazy}

\subsection{Prehľad zvyšnej časti dokumentu}

V časti \ref{general} popíšeme situáciu, do ktorej je systém zasadený (\ref{gen:perspective}), aké funkcie má poskytovať (\ref{gen:functions}), aké typy používateľov bude mať (\ref{gen:users}), aké existujúce postupy/procesy/predpisy v systéme vystupujú (\ref{gen:constraints}) a popis rozhrania systému s okolitým svetom, prípadne inými systémami (\ref{gen:deps}).

V časti \ref{reqs} uvedieme ucelený zoznam všetkých požiadaviek na systém.

\cleardoublepage
\section{Všeobecný popis}
\label{general}

\subsection{Perspektíva systému}
\label{gen:perspective}
Webové rozhranie bude slúžiť na evidenciu fľiaš s rôznymi plynmi a jej jednoduchej navigácií.

\subsection{Funkcie systému}
\label{gen:functions}
Systém bude schopný evidovať fľaše s rôznymi plynmi, spoločne s podrobnejšími informáciami o fľašiach ako sú typ plynu, miesto uskladnenia, aktuálna poloha a podobne. Pri každej fľaši bude aj priestor na pridávanie komentáru ku danej fľaši. Systém musí byť schopný zachovať históriu fliaš, či už vyradených alebo stále v obehu. Systém bude implementovaný ako webové rozhranie. Pridávanie nových fliaš je zabezpečené pomocou čiarových kódov ktorými disponuje každá fľaša. Všetky vlastnosti fliaš budú editovateľné používateľmi. Systém bude mať rôznych používateľov s rôznymi oprávneniami so zásahom do údajov. Systém bude optimalizovaný pre mobilné aj pre desktop prostredia. Systém bude disponovať aj anglickou verziou webového rozhrania. Dáta v systéme bude používateľ schopný filtrovať pomocou predom nastavených filtrov alebo vyhľadávacieho poľa. Tlak vo fľašiach bude pridávaní pomocou odfotenia stavu nanometra používateľom. Stav nanometra bude následne automaticky vyhodnotení systémom. Pri prípadnej zlej detekcii stavu nanometra bude údaj mať možnosť upravenia. Všetky údaje budú automaticky ukladané systémom.

\subsection{Charakteristika používateľov}
\label{gen:users}

\subsubsection{Používateľ s prístupom iba na čítanie}
\label{gen:users:ro}

Tento používateľ si môže pozerať všetky dáta v systéme, vyhľadávať a filtrovať flaše, prezerať históriu. K systému má prístup iba na čítanie, nemôže do údajov zasahovať ani ich meniť.

\subsubsection{Používateľ s prístupom na evidenciu tlaku}
\label{gen:users:meter}

Má všetky možnosti ako \ref{gen:users:ro}. Navyše si môže vybrať flašu alebo naskenovať čiarový kóď flaše a zadať pre danú flašu aktuálny stav tlaku manuálnym zadaním, alebo automatickým odčítaním z manometra.

\subsubsection{Používateľ s administrátorským prístupom}
\label{gen:users:admin}

Má všetky možnosti ako \ref{gen:users:meter}. Navyše môže upravovať všetky parametre fliaš, evidovať ich pozíciu, prijímať nové flaše do systému a vyraďovať flaše zo systému.

\subsection{Všeobecné obmedzenia}
\label{gen:constraints}

\subsection{Predpoklady a závislosti}
\label{gen:deps}

Systém bude vyvíjaný ako webová aplikácia pre stolné počítače a mobilné zariadenia. Aplikácia bude vyžadovať pripojenie k internetu. Aplikácia bude závisieť od servera, ku ktorému sa pripojí. Aplikácia predpokladá, že každá fľaša s plynom bude mať práve jeden unikátny a nepoškodený čiarový kód. Aplikácia bude vyžadovať prístup k fotoaparátu mobilného zariadenia pri pokuse o skenovanie čiarového kódu.

\cleardoublepage
\section{Špecifické požiadavky}
\label{reqs}

\cleardoublepage
\section{Prílohy}

\end{document}
