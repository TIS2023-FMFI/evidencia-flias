\documentclass[hreffootnote]{zah}

\title{Evidencia plynových fliaš}
\author{Oliver Laštík, Šimon Strieška, Jozef Špirka, Adam Zahradník}

\begin{document}
\maketitle

\tableofcontents
\cleardoublepage

\section{Úvod}

\subsection{Účel tohto dokumentu}

Tento dokument slúži ako komplexný opis systému, ktorého cieľom je poskytnúť súbor požiadaviek pre navrhovaný systém. Slúži vývojovému tímu ako jednotné miesto, v ktorom sú spísané všetky požiadavky na systém. Zároveň slúži pre zadávateľa pre prehľad dohodnutých požiadavok a kontrolu ich naplnenia.

\subsection{Rozsah pôsobnosti systému}

Systém je určený na evidenciu používaných plynových fliaš, zaznamenávanie ich umiestnenia vo forme skladovej evidencie a sledovanie spotrebu plynov prostredníctvom "odpisovania" tlaku vo flašiach.

\subsection{Používané definície a skratky}

\begin{enumerate}
	\item manometer: prístroj na meranie tlaku, ciferník s ručičkou
\end{enumerate}

\subsection{Odkazy}

\subsection{Prehľad zvyšnej časti dokumentu}

V časti \ref{general} popíšeme situáciu, do ktorej je systém zasadený (\ref{gen:perspective}), aké funkcie má poskytovať (\ref{gen:functions}), aké typy používateľov bude mať (\ref{gen:users}), aké existujúce postupy/procesy/predpisy v systéme vystupujú (\ref{gen:constraints}) a popis rozhrania systému s okolitým svetom, prípadne inými systémami (\ref{gen:deps}).

V časti \ref{reqs} uvedieme ucelený zoznam všetkých požiadaviek na systém.

\cleardoublepage
\section{Všeobecný popis}
\label{general}

\subsection{Perspektíva systému}
\label{gen:perspective}

\subsection{Funkcie systému}
\label{gen:functions}

\subsection{Charakteristika používateľov}
\label{gen:users}

\subsubsection{Používateľ s prístupom iba na čítanie}
\label{gen:users:ro}

Tento používateľ si môže pozerať všetky dáta v systéme, vyhľadávať a filtrovať flaše, prezerať históriu. K systému má prístup iba na čítanie, nemôže do údajov zasahovať ani ich meniť.

\subsubsection{Používateľ s prístupom na evidenciu tlaku}
\label{gen:users:meter}

Má všetky možnosti ako \ref{gen:users:ro}. Navyše si môže vybrať flašu alebo naskenovať čiarový kóď flaše a zadať pre danú flašu aktuálny stav tlaku manuálnym zadaním, alebo automatickým odčítaním z manometra.

\subsubsection{Používateľ s administrátorským prístupom}
\label{gen:users:admin}

Má všetky možnosti ako \ref{gen:users:meter}. Navyše môže upravovať všetky parametre fliaš, evidovať ich pozíciu, prijímať nové flaše do systému a vyraďovať flaše zo systému.

\subsection{Všeobecné obmedzenia}
\label{gen:constraints}

\subsection{Predpoklady a závislosti}
\label{gen:deps}

Systém bude vyvíjaný ako webová aplikácia pre stolné počítače a mobilné zariadenia. Aplikácia bude vyžadovať pripojenie k internetu. Aplikácia bude závisieť od servera, ku ktorému sa pripojí. Aplikácia predpokladá, že každá fľaša s plynom bude mať práve jeden unikátny a nepoškodený čiarový kód. Aplikácia bude vyžadovať prístup k fotoaparátu mobilného zariadenia pri pokuse o skenovanie čiarového kódu.

\cleardoublepage
\section{Špecifické požiadavky}
\label{reqs}

\cleardoublepage
\section{Prílohy}

\end{document}
