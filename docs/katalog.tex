\documentclass[hreffootnote]{zah}

\title{Evidencia plynových fliaš}
\author{Oliver Laštík, Šimon Strieška, Jozef Špirka, Adam Zahradník}

\begin{document}
\maketitle

\tableofcontents
\cleardoublepage

\section{Úvod}

\subsection{Účel tohto dokumentu}

Tento dokument slúži ako komplexný opis systému, ktorého cieľom je poskytnúť súbor požiadaviek pre navrhovaný systém. Slúži vývojovému tímu ako jednotné miesto, v ktorom sú spísané všetky požiadavky na systém. Zároveň slúži pre zadávateľa pre prehľad dohodnutých požiadavok a kontrolu ich naplnenia.

\subsection{Rozsah pôsobnosti systému}

Systém je určený na evidenciu používaných plynových fliaš, zaznamenávanie ich umiestnenia vo forme skladovej evidencie a sledovanie spotrebu plynov prostredníctvom "odpisovania" tlaku vo fľašiach.

\subsection{Používané definície a skratky}

\begin{enumerate}
	\item manometer: prístroj na meranie tlaku, ciferník s ručičkou
 	\item MP: Megapixel, milión pixelov (malé body tvoriace časť obrázka na obrazovke počítača), používaný na meranie množstva detailov v obrazoch vytvorených digitálnym fotoaparátom, obrazovkou počítača atď.
 	\item čítačka čiarový kódov s emuláciou klávesnice: čítačka čiarových kódov, ktorá naskenované kódy vypisuje na klávesnicu pripojeného počítača
\end{enumerate}

\subsection{Odkazy}

\subsection{Prehľad zvyšnej časti dokumentu}

V časti \ref{general} popíšeme situáciu, do ktorej je systém zasadený (\ref{gen:perspective}), aké funkcie má poskytovať (\ref{gen:functions}), aké typy používateľov bude mať (\ref{gen:users}), aké existujúce postupy/procesy/predpisy v systéme vystupujú (\ref{gen:constraints}) a popis rozhrania systému s okolitým svetom, prípadne inými systémami (\ref{gen:deps}).

V časti \ref{reqs} uvedieme ucelený zoznam všetkých požiadaviek na systém.

\cleardoublepage
\section{Všeobecný popis}
\label{general}

\subsection{Perspektíva systému}
\label{gen:perspective}
Systém bude implementovaný ako webové rozhranie ktoré búde slúžiť k evidencii fľiaš s plynmi ktorých parametre budú uchovávané v databáze. Bude slúžiť ako uľahčenie evidencie fľiaš v laboratóriach na fakulte matematiky,fyziky a informatiky UK. Fľaše majú rôzne parametre ktoré užívateľ môže zadať pre ich jednoduché odlíšenie. Systém disponuje aj filtrom  kvôli jednoduchej navigácií v postredí systému. Taktiež uživatelia si budú môcť pozrieť pri jednotlivých fľašiach aj ich históriu pre jednoduché dohľadanie potrebných údajov. Programovací jazyk použitý pri tvorbe bude Python, presnejšie framework Django s prvkami HTML/CSS/JS.

\subsection{Funkcie systému}
\label{gen:functions}
Systém bude schopný evidovať fľaše s rôznymi plynmi, spoločne s podrobnejšími informáciami o fľašiach ako sú typ plynu, miesto uskladnenia, aktuálna poloha a podobne. Pri každej fľaši bude aj priestor na pridávanie komentáru ku danej fľaši. Systém musí byť schopný zachovať históriu fliaš ako naplnenie fliaš, pohyb fliaš, status fliaš(vyradená, pripojená, vyskladnená, vyradená), umiestnenie a tak isto aj ďalšie parametre. Systém bude implementovaný ako webové rozhranie. Každá fľaša disponuje svojim jednoznačným číselným kódom, ktorý je reprezentovaný ako čiarový kód načitateľný systémom. Všetky vlastnosti fliaš budú editovateľné používateľmi. Systém bude mať používateľov s rôznym prístupom k systému (Čitateľ, Editor, Administrátor). Systém bude optimalizovaný pre mobilné aj pre desktop prostredia. Systém bude disponovať aj anglickou verziou webového rozhrania. Dáta v systéme bude používateľ schopný filtrovať pomocou predom nastavených filtrov, ako napríklad typ plynu, miestnosť, status fľaše a podobne a taktiež systém bude disponovať aj full-textovým vyhľadávaním. Tlak vo fľašiach bude pridávaní pomocou odfotenia stavu manometra používateľom alebo manuálnym zadaním údajov. Pri odfotení bude jeho stav automaticky vyhodnotení systémom. Pri prípadnej zlej detekcie stavu manometra bude možnosť manuálneho upravenia.

\subsection{Charakteristika používateľov}
\label{gen:users}

Systém podporuje nasledujúce druhy používateľov. Môže existovať viacero rôznych používateľov rovnakého druhu.

\subsubsection{Používateľ s prístupom iba na čítanie (Čitateľ)}
\label{gen:users:ro}

Tento používateľ si môže pozerať všetky dáta v systéme, vyhľadávať a filtrovať fľaše, prezerať históriu. K systému má prístup iba na čítanie, nemôže do údajov zasahovať ani ich meniť.

\subsubsection{Používateľ s prístupom na evidenciu tlaku (Editor)}
\label{gen:users:meter}

Má všetky možnosti ako \ref{gen:users:ro}. Navyše si môže vybrať fľašu alebo naskenovať čiarový kóď fľaše a zadať pre danú fľašu aktuálny stav tlaku manuálnym zadaním, alebo automatickým odčítaním z manometra.

\subsubsection{Používateľ s administrátorským prístupom (Administrátor)}
\label{gen:users:admin}

Má všetky možnosti ako \ref{gen:users:meter}. Navyše môže upravovať všetky parametre fliaš, evidovať ich pozíciu, prijímať nové fľaše do systému, vyraďovať fľaše zo systému, pridávať a upravovať používateľov do systému.

\subsection{Všeobecné obmedzenia}
\label{gen:constraints}

Medzi obmedzenia systému patrí prístup k internetu a fotoaparát s kamerou aspoň 5 MP pre zachovanie dostatočnej kvality fotografie na čítanie hodnôt na manometri. 

\subsection{Predpoklady a závislosti}
\label{gen:deps}

Systém bude vyvíjaný ako webová aplikácia pre stolné počítače a mobilné zariadenia. Systém bude vyžadovať pripojenie k internetu. Systém bude závisieť od servera, ku ktorému sa pripojí. Systém predpokladá, že každá fľaša s plynom bude mať práve jeden unikátny a nepoškodený čiarový kód. Systém bude vyžadovať prístup k fotoaparátu mobilného zariadenia pri pokuse o skenovanie čiarového kódu.

\cleardoublepage
\section{Špecifické požiadavky}
\label{reqs}

\begin{enumerate}
\item Administrátor môže do systému pridať nových používateľov, pričom im nastaví meno, email, heslo a druh prístupu (\ref{gen:users}).
\item Administrátor môže deaktivovať existujúcich používateľov.
\item Deaktivovaný používateľ sa nemôže prihlásiť, ale ním vykonané zmeny (\ref{req:history}) sa v systéme zachovajú.
\item Používateľ sa do systému prihlasuje pomocou emailu a hesla.
\item Používateľ môže zmeniť svoje heslo po prihlásení.
\item Používateľ môže (bez prihlásenia) požiadať o zaslanie emailu na obnovu hesla.
\item Administrátor môže meniť meno, email, heslo a druh prístupu (\ref{gen:users}) používateľom.
\item Systém pre každú fľašu eviduje nasledujúce parametre: 
\begin{enumerate}
	\tightlist
	\item čiarový kód (text, unikátny identifikátor)
	\item \label{req:param:owner} majiteľ (výber z možností)
	\item \label{req:param:provider} dodávateľ (výber z možností)
	\item \label{req:param:gas} plyn (výber z možností)
	\item čistota (desatinné číslo)
	\item objem (číslo)
	\item \label{req:param:pressure} tlak (desatinné číslo)
	\item dátum prevzatia (dátum)
	\item dátum zapojenia (dátum)
	\item dátum odovzdania (dátum)
	\item poznámka (text)
	\item \label{req:param:location} umiestnenie (výber z možností)
	\item stav (výber z možností - "prijatá", "vrátená", "vyradená")
\end{enumerate}
\item Administrátor môže vytvoriť novú fľašu. Môže ju vytvoriť aj bez uvedenia všetkých parametrov.
\item Administrátor môže kedykoľvek upravovať hodnoty parametrov fľaše.
\item Administrátor môže pridávať, upravovať a mazať možnosti pre parametre "majiteľ" (\ref{req:param:owner}), "dodávateľ" (\ref{req:param:provider}), "plyn" (\ref{req:param:gas}), "umiestnenie" (\ref{req:param:location}).
\item \label{req:location_details} Administrátor môže umiestneniu fľaše nastaviť budovu, pracovisko a zodpovednú osobu.
\item Administrátor môže pridávať, upravovať a mazať možnosti pre "budova", "pracovisko" a "zodpovedná osoba" (\ref{req:location_details}).
\item \label{req:history} Systém ukladá kompletnú históriu zmien parametrov o fľaši.
\item Čitateľ môže vyhľadávať v zozname fľaší pomocou fulltextového vyhľadávania.
\item Čitateľ môže filtrovať v zozname fľaší podľa jednotlivých parametrov a ich kombinácií.
\item Čitateľ si môže pozerať históriu zmien parametrov konkrétnej fľaše.
\item Čitateľ si môže zobraziť históriu zmien parametru "tlak" (\ref{req:param:pressure}) konkrétnej fľaše na čiarovom grafe.
\item Čitateľ si môže zobraziť informácie o konkrétnej fľaši ručným vyhľadaním v systéme alebo naskenovaním jej čiarového kódu pomocou mobilného telefónu alebo čítačky čiarového kódu s emuláciou klávesnice.
\item Editor môže zaevidovať stav tlaku danej fľaši manuálnym zadaním, nahraním fotografie manometra alebo odfotením manometra.
Systém z poskytnutej fotografie odčíta aktuálny stav manometra. Používateľ môže skontrolovať správnosť hodnoty a v prípade potreby ju upraviť.
\item Editor môže pri poskytovaní fotografie manometra označiť na ciferníku manometra minimálnu (nulovú) a maximálnu hodnotu. Tieto hodnoty potom systém použije pri výpočte aktuálnej hodnoty podľa polohy ručicky.
\end{enumerate}

\subsection{Požiadavky nevzťahujúce sa na funkcionalitu}

\begin{enumerate}
\item Používateľ môže k systému pristupovať z mobilného zariadenia alebo počítača.
\item Používateľ si môže prepínať medzi anglickou a slovenskou verziou jazyka.
\end{enumerate}


\cleardoublepage
\section{Prílohy}

\end{document}
