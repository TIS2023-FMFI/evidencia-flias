\documentclass[hreffootnote]{zah}

\title{Evidencia plynových fliaš}
\author{Oliver Laštík, Šimon Strieška, Jozef Špirka, Adam Zahradník}

\begin{document}
\maketitle

\tableofcontents
\cleardoublepage

\section{Úvod}

\subsection{Účel tohto dokumentu}

Tento dokument slúži ako komplexný opis systému, ktorého cieľom je poskytnúť súbor požiadaviek pre navrhovaný systém. Slúži vývojovému tímu ako jednotné miesto, v ktorom sú spísané všetky požiadavky na systém. Zároveň slúži pre zadávateľa pre prehľad dohodnutých požiadavok a kontrolu ich naplnenia.

\subsection{Rozsah pôsobnosti systému}

Systém je určený na evidenciu používaných plynových fliaš, zaznamenávanie ich umiestnenia vo forme skladovej evidencie a sledovanie spotrebu plynov prostredníctvom "odpisovania" tlaku vo fľašiach.

\subsection{Používané definície a skratky}

\begin{enumerate}
	\item manometer: prístroj na meranie tlaku, ciferník s ručičkou
 	\item MP: Megapixel, milión pixelov (malé body tvoriace časť obrázka na obrazovke počítača), používaný na meranie množstva detailov v obrazoch vytvorených digitálnym fotoaparátom, obrazovkou počítača atď.
 	\item čítačka čiarový kódov s emuláciou klávesnice: čítačka čiarových kódov, ktorá naskenované kódy vypisuje na klávesnicu pripojeného počítača
\end{enumerate}

\subsection{Odkazy}

\subsection{Prehľad zvyšnej časti dokumentu}

V časti \ref{general} popíšeme situáciu, do ktorej je systém zasadený (\ref{gen:perspective}), aké funkcie má poskytovať (\ref{gen:functions}), aké typy používateľov bude mať (\ref{gen:users}), aké existujúce postupy/procesy/predpisy v systéme vystupujú (\ref{gen:constraints}) a popis rozhrania systému s okolitým svetom, prípadne inými systémami (\ref{gen:deps}).

V časti \ref{reqs} uvedieme ucelený zoznam všetkých požiadaviek na systém.

\cleardoublepage
\section{Všeobecný popis}
\label{general}

\subsection{Perspektíva systému}
\label{gen:perspective}
Systém bude implementovaný ako webové rozhranie ktoré búde slúžiť k evidencii fľiaš s plynmi ktorých parametre budú uchovávané v databáze. Bude slúžiť ako uľahčenie evidencie fľiaš v laboratóriach na fakulte matematiky,fyziky a informatiky UK. Fľaše majú rôzne parametre ktoré užívateľ môže zadať pre ich jednoduché odlíšenie. Systém disponuje aj filtrom  kvôli jednoduchej navigácií v postredí systému. Taktiež uživatelia si budú môcť pozrieť pri jednotlivých fľašiach aj ich históriu pre jednoduché dohľadanie potrebných údajov. Programovací jazyk použitý pri tvorbe bude Python, presnejšie framework Django s prvkami HTML/CSS/JS.

\subsection{Funkcie systému}
\label{gen:functions}
Systém bude schopný evidovať fľaše s rôznymi plynmi, spoločne s podrobnejšími informáciami o fľašiach ako sú typ plynu, miesto uskladnenia, aktuálna poloha a podobne. Pri každej fľaši bude aj priestor na pridávanie komentáru ku danej fľaši. Systém musí byť schopný zachovať históriu fliaš ako naplnenie fliaš, pohyb fliaš, status fliaš(vyradená, pripojená, vyskladnená, vyradená), umiestnenie a tak isto aj ďalšie parametre. Systém bude implementovaný ako webové rozhranie. Každá fľaša disponuje svojim jednoznačným číselným kódom, ktorý je reprezentovaný ako čiarový kód načitateľný systémom. Všetky vlastnosti fliaš budú editovateľné používateľmi. Systém bude mať používateľov s rôznym prístupom k systému (Čitateľ, Editor, Administrátor). Systém bude optimalizovaný pre mobilné aj pre desktop prostredia. Systém bude disponovať aj anglickou verziou webového rozhrania. Dáta v systéme bude používateľ schopný filtrovať pomocou predom nastavených filtrov, ako napríklad typ plynu, miestnosť, status fľaše a podobne a taktiež systém bude disponovať aj full-textovým vyhľadávaním. Tlak vo fľašiach bude pridávaní pomocou odfotenia stavu manometra používateľom alebo manuálnym zadaním údajov. Pri odfotení bude jeho stav automaticky vyhodnotení systémom. Pri prípadnej zlej detekcie stavu manometra bude možnosť manuálneho upravenia.

\subsection{Charakteristika používateľov}
\label{gen:users}

Systém podporuje nasledujúce druhy používateľov. Môže existovať viacero rôznych používateľov rovnakého druhu.

\subsubsection{Používateľ s prístupom iba na čítanie (Čitateľ)}
\label{gen:users:ro}

Tento používateľ si môže pozerať všetky dáta v systéme, vyhľadávať a filtrovať fľaše, prezerať históriu. K systému má prístup iba na čítanie, nemôže do údajov zasahovať ani ich meniť.

\subsubsection{Používateľ s prístupom na evidenciu tlaku (Editor)}
\label{gen:users:meter}

Má všetky možnosti ako \ref{gen:users:ro}. Navyše si môže vybrať fľašu alebo naskenovať čiarový kóď fľaše a zadať pre danú fľašu aktuálny stav tlaku manuálnym zadaním, alebo automatickým odčítaním z manometra.

\subsubsection{Používateľ s administrátorským prístupom (Administrátor)}
\label{gen:users:admin}

Má všetky možnosti ako \ref{gen:users:meter}. Navyše môže upravovať všetky parametre fliaš, evidovať ich pozíciu, prijímať nové fľaše do systému, vyraďovať fľaše zo systému, pridávať a upravovať používateľov do systému.

\subsection{Všeobecné obmedzenia}
\label{gen:constraints}

Medzi obmedzenia systému patrí prístup k internetu a fotoaparát s kamerou aspoň 5 MP pre zachovanie dostatočnej kvality fotografie na čítanie hodnôt na manometri. 

\subsection{Predpoklady a závislosti}
\label{gen:deps}

Systém bude vyvíjaný ako webová aplikácia pre stolné počítače a mobilné zariadenia. Systém bude vyžadovať pripojenie k internetu. Systém bude závisieť od servera, ku ktorému sa pripojí. Systém predpokladá, že každá fľaša s plynom bude mať práve jeden unikátny a nepoškodený čiarový kód. Systém bude vyžadovať prístup k fotoaparátu mobilného zariadenia pri pokuse o skenovanie čiarového kódu.

\cleardoublepage
\section{Špecifické požiadavky}

\begin{enumerate}
\item Admin môže do systému pridať nových používateľov, pritom nastaví ich e-mail, heslo a úroveň prístupu.
\item Admin dokáže deaktivovať existujúcich používateľov, stratia schopnosť prihlásiť sa, no ich účet a história ich aktivity bude naďalej existovať.
\item Používateľ sa do systému prihlasuje pomocou e-mailu a hesla.
\item Používateľ môže zmeniť svoje heslo po prihlásení alebo žiadosťou o zmenu hesla zaslaním formulára na e-mail.
\item Použivateľ si môže heslo resetnúť pomocou svojho e-mailu.
\item Admin môže meniť práva ostatných používateľov.
\item Admin môže deaktiovať účty ostatných používateľov.
\item Systém eviduje plynové fľaše podľa ich čiarového kódu.
\item Systém pre každú fľašu eviduje nasledujúce parametre: majiteľ, čiarový kód, dodávateľ, plyn, čistota, objem, tlak, dátum prevzatia, dátum zapojenia, umiestnenie, dátum odovzdania, poznámka.
\item Systém by mal byť postavený tak, aby bolo možné jednoducho pridať nové parametre.
\item Každá fľaša je buď v sklade, pripojená, vyskladnená alebo vyradená.
\item Systém ukladá kompletnú históriu zmien parametrov o fľaši.
\item Admin môže upravovať hodnoty parametrov fľaše.
\item Umiestnenie fľaše by malo podporovať štuktúru - budova, miestnosť, zariadenie.
\item Umiestnenie fľaše má pridelenú zodpovednú osobu.
\item Admin môže upravovať a pridávať umiestnenia.
\item Používateľ má možnosť vyhľadávať fľaše pomocou fulltext searchu.
\item Používateľ má možnosť filtrovať fľaše podľa jednotlivých parametrov a ich kombinácií.
\item Používateľ má možnosť pozerať si historické dáta a históriu jednotlivých fliaš.
\item Systém zobrazuje stav tlaku v čase pomocou grafov.
\item Editor môže naskenovať čiarový kód fľaše čítačkou s emuláciou klávesnice alebo pomocou kamery mobilného telefónu na rýchly výber fľaše.
\item Editor môže upraviť aktuálny stav tlaku jednotlivým fľašiam.
\item Editor môže odfotiť manometer na fľaši, z fotografie manometra systém rozpozná aktuálny tlak.
\item Editor môže nakalibrovať manometer označením minimálnej a maximálnej hodnoty na manometri.
\item Po skontrolovaní správnosti detekcie používateľom systém uloží stav tlaku.
\item Editor môže v prípade nesprávnej automatickej detekcie tlak upraviť ručne.
\item Systém je dostupný z počítača aj mobilných zariadení.
\item Systém je dostupný v anglickom a slovenskom jazyku.
\item Používateľ si môže medzi jazykmi prepínať.
\end{enumerate}
\subsection{Požiadavky nevzťahujúce sa na funkcionalitu}

\label{reqs}

\cleardoublepage
\section{Prílohy}

\end{document}
